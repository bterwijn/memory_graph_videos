\documentclass[10pt, colorlinks=true, urlcolor=blue]{beamer}

% Themes and colors for a professional look
\usetheme{Madrid}
\usecolortheme{seagull}
\setbeamercolor{title}{fg=white,bg=blue!80!black}
\setbeamercolor{frametitle}{fg=white,bg=blue!70!black}
\setbeamercolor{block title}{fg=white,bg=blue!80!black}
\setbeamercolor{block body}{bg=blue!5!white}
\setbeamerfont{title}{size=\fontsize{12}{16}\selectfont}
\setlength{\parindent}{0pt}

% Customize the footline layout
\setbeamertemplate{footline}
{
  \leavevmode%
  \hbox{%
    % Left field (author or custom text)
    \begin{beamercolorbox}[wd=0.25\paperwidth,ht=2.5ex,dp=1ex,center]{author in head/foot}%
      \usebeamerfont{author in head/foot}\insertshortauthor
    \end{beamercolorbox}%
    % Middle field (title)
    \begin{beamercolorbox}[wd=0.5\paperwidth,ht=2.5ex,dp=1ex,center]{title in head/foot}%
      \usebeamerfont{title in head/foot}\insertshorttitle
    \end{beamercolorbox}%
    % Right field (page number)
    \begin{beamercolorbox}[wd=0.25\paperwidth,ht=2.5ex,dp=1ex,center]{date in head/foot}%
      \usebeamerfont{date in head/foot}\insertframenumber/\inserttotalframenumber
    \end{beamercolorbox}%
  }%
}

% Set hyperlink colors
\hypersetup{
    colorlinks=true,    % Enable colored links
    urlcolor=blue,      % Color for URLs
    linkcolor=blue      % Color for internal links (optional)
}

\usepackage{minted}
\usepackage{tcolorbox}
% enable minted support and breakable boxes
% cbuselibrary{minted,skins,breakable}

\newcommand{\listfile}[2][]{
  \begin{tcolorbox}[
      title=#2,
      fonttitle=\tiny,
      boxsep=0pt,
      top=2pt,
      bottom=2pt,
      left=1pt,
      right=0pt,
      toptitle=0mm,
      bottomtitle=0mm,
      halign title=center,
      box align = center,
      nobeforeafter]
      \inputminted[#1]{python}{#2}\end{tcolorbox}
}


\title{Memory\_Graph, a Python Teaching Tool and Debugging Aid}
\author{Bas Terwijn}
\titlegraphic{
\vspace{-2.8em} % Adjust this value as needed
  \makebox[\textwidth]{%
  \begin{tabular}{c c}
    \includegraphics[height=0.35\textwidth]{figures/immutable.png} &
    \includegraphics[height=0.35\textwidth]{figures/mutable.png} \\
      &  \\
    \multicolumn{2}{c}{
      \includegraphics[width=0.6\textwidth]{figures/uva.png}
    }
  \end{tabular}
}
}
\date{}

\begin{document}

\begin{frame}
    \titlepage
\end{frame}

\begin{frame}{Topics}
  Teaching Tool: students learn the right mental model for Python data \\
  Debugging Aid: students fix bugs after visualizing/understanding data \\
  \\
  Memory\_Graph, a modern version of \href{https://pythontutor.com/}{Python Tutor} (25 million users)\\
  \\
  Topics:
  \begin{enumerate}
    \item Explain Python Data Model (to demo Memory\_Graph)
    \item Data Model Exercises
    \item Code Examples
    \item Three Usage Modes
    \item Configuration
    \item Future Work: Debugging in Production Code
  \end{enumerate}
\end{frame}

\begin{frame}{Python Data Model}
  \begin{block}{Python Types}
    Python has two distinct categories of types: Immutable, Mutable
  \end{block}
  
  \vspace{2.4em}
  
  \textbf{Immutable} Types: \texttt{bool}, \texttt{int}, \texttt{float}, \texttt{complex}, \texttt{str}, \texttt{tuple}, \texttt{bytes}, \texttt{frozenset} \\
  
  \vspace{-0.8em}
  A value of an immutable type \textbf{cannot} be mutated in place. \\
  So when it is changed, \textbf{an} automatic copy is made. \\
  
  \vspace{2.0em}
  
  \textbf{Mutable} Types: \texttt{list}, \texttt{set}, \texttt{dict}, \texttt{classes}, \dots (most other types) \\
  
  \vspace{-0.8em}
  A value of a mutable type \textbf{can} be mutated in place. \\
  So when it is changed, \textbf{no} copy is made.
\end{frame}


\begin{frame}[fragile]{Mutability, live in ipython}
  \vspace{-33.5mm}

  \begin{tcolorbox}[boxsep=0pt,left=2pt,right=2pt,top=2pt,bottom=2pt]
  \begin{minted}[fontsize=\small]{python}
  tuple1 = (4, 3, 2)  # tuple, immutable
  tuple2 = tuple1
  tuple2 += (1,)

  list1 = [4, 3, 2]   # list, mutable
  list2 = list1
  list2 += [1]
  \end{minted}
  \end{tcolorbox}

\end{frame}

\begin{frame}[fragile]{Copying, live in Web Debugger}
  So, values of immutable type are copied automatically when changed, values of mutable type are not.
  \vspace{2em}
  
  To copy mutable values ourself, Python has three options:
 \href{https://memory-graph.com/#code=import%20copy%0A%0Aa%20%3D%20%5B%20%5B1%2C%202%5D%2C%20%5B'x'%2C%20'y'%5D%20%5D%0A%0Ac1%20%3D%20a%20%20%20%20%20%20%20%20%20%20%20%20%20%20%20%20%0Ac2%20%3D%20copy.copy(a)%20%20%20%20%20%23%20or%3A%20%20a.copy()%20%20list(a)%20%20a%5B%3A%5D%0Ac3%20%3D%20copy.deepcopy(a)%20%0A%0A%23%20c1%3A%20assignment%2C%20%20%20nothing%20is%20copied%2C%20everything%20is%20shared%0A%23%20c2%3A%20shallow%20copy%2C%20first%20element%20is%20copied%2C%20underlying%20is%20shared%0A%23%20c3%3A%20deep%20copy%2C%20%20%20%20everything%20is%20copied%2C%20nothing%20is%20shared%0A}{Copying}
 \vspace{2em}

 Or define your own custom copy logic:
 \href{https://memory-graph.com/#code=import%20copy%0A%0Aa%20%3D%20%5B%20%5B1%2C%202%5D%2C%20%5B'x'%2C%20'y'%5D%20%5D%0A%0Ac1%20%3D%20a%20%20%20%20%20%20%20%20%20%20%20%20%20%20%20%20%0Ac2%20%3D%20copy.copy(a)%20%20%20%20%20%23%20or%3A%20%20a.copy()%20%20list(a)%20%20a%5B%3A%5D%0Ac3%20%3D%20copy.deepcopy(a)%20%0A%0A%23%20c1%3A%20assignment%2C%20%20%20nothing%20is%20copied%2C%20everything%20is%20shared%0A%23%20c2%3A%20shallow%20copy%2C%20first%20element%20is%20copied%2C%20underlying%20is%20shared%0A%23%20c3%3A%20deep%20copy%2C%20%20%20%20everything%20is%20copied%2C%20nothing%20is%20shared%0A%0Adef%20custom_copy(a)%3A%0A%20%20%20%20c%20%3D%20a.copy()%20%23%20shallow%20copy%0A%20%20%20%20c%5B1%5D%20%3D%20a%5B1%5D.copy()%0A%20%20%20%20return%20c%0A%20%20%20%20%0Ac4%20%3D%20custom_copy(a)%0A%0A%23%20c4%3A%20custom%20copy%2C%20%20%20you%20decide%20what%20is%20copied%20and%20shared%0A&breakpoints=18}{Custom Copy}
\end{frame}

\begin{frame}[fragile]{Name Rebinding, live in Web Debugger}
  Difference between changing a variable and reassigning it:
  \href{https://memory-graph.com/#code=%0Aa+%3D+%5B4%2C+3%2C+2%5D%0Ab+%3D+a%0A%0Ab+%2B%3D+%5B1%5D++++++++%23+changing+%27b%27+changes+%27a%27%0Ab+%3D+%5B100%2C+200%5D++%23+but+reassignment+rebinds+%27b%27+to+another+value%2C+%27a%27+is+uneffected%0A}{Name Rebinding}
\end{frame}

\begin{frame}[fragile]{Function call, live in Web Debugger}
  Pass by object reference:
  \href{https://memory-graph.com/#codeurl=https://raw.githubusercontent.com/bterwijn/memory_graph/refs/heads/main/src/function_call.py}{Function Call with Arguments}
\end{frame}

\begin{frame}{Data Model Exercises}
  Multiple Choice Exercises, limited time, your best guess please:
  \begin{itemize}
  \item \href{https://memory-graph.com/\#codeurl=https://raw.githubusercontent.com/bterwijn/memory_graph_videos/refs/heads/main/exercises/exercise3.py}{Exercise 3}
  \item \href{https://memory-graph.com/\#codeurl=https://raw.githubusercontent.com/bterwijn/memory_graph_videos/refs/heads/main/exercises/exercise8.py}{Exercise 8}
  \item \href{https://memory-graph.com/\#codeurl=https://raw.githubusercontent.com/bterwijn/memory_graph_videos/refs/heads/main/exercises/exercise9.py}{Exercise 9}
  \item \href{https://memory-graph.com/\#codeurl=https://raw.githubusercontent.com/bterwijn/memory_graph_videos/refs/heads/main/exercises/exercise5.py}{Exercise 5}
  \end{itemize}
  \vspace{8mm}
  Or see \href{https://github.com/bterwijn/memory_graph_videos/blob/main/exercises/exercises.md}{more exercises}.
\end{frame}

\begin{frame}[fragile]{Code Examples}
  Common exercises for Python beginners:
  
  \href{https://memory-graph.com/#codeurl=https://raw.githubusercontent.com/bterwijn/memory_graph/refs/heads/main/src/value_counts.py}{Count the Values using a dict}
  
\end{frame}

\begin{frame}[fragile]{Recursion Example}
\begin{verbatim}
factorial(n) = n * factorial(n-1)     if n > 1
             = 1                      otherwise

factorial(4) = 4 * factorial(3)
             = 4 * 3 * factorial(2)
             = 4 * 3 * 2 * factorial(1)
             = 4 * 3 * 2 * 1 * factorial(0)
             = 4 * 3 * 2 * 1 * 1
             = 4 * 3 * 2 * 1
             = 4 * 3 * 2
             = 4 * 6
             = 24
\end{verbatim}
\begin{itemize}
\item \href{https://memory-graph.com/\#codeurl=https://raw.githubusercontent.com/bterwijn/memory_graph/refs/heads/main/src/factorial.py}{factorial}
\item \href{https://invocation-tree.com/\#codeurl=https://raw.githubusercontent.com/bterwijn/memory_graph/refs/heads/main/src/factorial.py}{factorial in Invocation Tree}
\end{itemize}

\end{frame}

\begin{frame}{Data Structure Example}
Larger Data Structures:
\begin{itemize}
\item \href{https://memory-graph.com/\#codeurl=https://raw.githubusercontent.com/bterwijn/memory_graph/refs/heads/main/src/linked_list.py&timestep=0.2&play}{Linked List}
\item \href{https://memory-graph.com/\#codeurl=https://raw.githubusercontent.com/bterwijn/memory_graph/refs/heads/main/src/bin_tree.py&timestep=0.2&play}{Binary Tree} (recursion)
\item \href{https://memory-graph.com/\#codeurl=https://raw.githubusercontent.com/bterwijn/memory_graph/refs/heads/main/src/hash_set.py&timestep=0.2&play}{Hash Set}
\item \href{https://memory-graph.com/\#codeurl=https://raw.githubusercontent.com/bterwijn/memory_graph/refs/heads/main/src/hash_map.py&timestep=0.2&play}{Hash Map}
\end{itemize}
\end{frame}

\begin{frame}{Sorting Examples}
Sorting (Algorithm) Examples:
\begin{itemize}
  \item \href{https://memory-graph.com/\#codeurl=https://raw.githubusercontent.com/bterwijn/memory_graph/refs/heads/main/src/selection_sort.py&breakpoints=13,27&continues=1&timestep=0.2}{Selection Sort}
  \item \href{https://memory-graph.com/\#codeurl=https://raw.githubusercontent.com/bterwijn/memory_graph/refs/heads/main/src/insertion_sort.py&breakpoints=13,29&continues=1&timestep=0.2}{Insertion Sort}
  \item \href{https://memory-graph.com/\#codeurl=https://raw.githubusercontent.com/bterwijn/memory_graph/refs/heads/main/src/bubble_sort.py&breakpoints=29,38&continues=1&timestep=0.2}{Bubble Sort}
  \item \href{https://memory-graph.com/\#codeurl=https://raw.githubusercontent.com/bterwijn/memory_graph/refs/heads/main/src/cocktail_sort.py&breakpoints=16,45&continues=1&timestep=0.2}{Cocktail Shaker Sort}
\end{itemize}
\end{frame}

\begin{frame}{Three Usage Modes: 1 Web Debugger}
  Memory Graph Web Debugger \\
  Copy code to \href{https://memory-graph.com/}{memory-graph.com} and run, no installation required. \\
  \\
  Some limitations:
  \begin{itemize}
    \item Only single file.
    \item No \mintinline{python}{input()} function (yet).
    \item Can't read/write files.
    \item ...
  \end{itemize}
\end{frame}

\begin{frame}[fragile]{Three Usage Modes: 2 Add Debug Graph Statements}
  Normal Python code.\\
  \\
  Quickly add "debug graph statements" where you want to better understand your code (similar to debug print statements):
  \begin{enumerate}
    \item add \mintinline{python}{import memory_graph as mg}
    \item add \mintinline{python}{mg.l()} alias for \mintinline{python}{mg.block(mg.show, locals())}
  \end{enumerate}
  \vspace{1em}
  \listfile[fontsize=\scriptsize,baselinestretch=1.4]{src/my\string_list.py}
\end{frame}

\begin{frame}{Three Usage Modes: 3 Integrated in IDE/enviroment}
  IDE Integration:
  \begin{itemize}
    \item Visual Studio Code (\href{https://raw.githubusercontent.com/bterwijn/memory_graph/main/images/vscode_copying.gif}{demo}, see \href{https://www.youtube.com/watch?v=23_bHcr7hqo}{instructions})
    \item Cursor AI
    \item PyCharm
    \item ...
  \end{itemize}
  Works through a 'watch' now, we are working on proper plugins.\\
  \vspace{1em}
  
  Environments:
  \begin{itemize}
    \item ipython
    \item Jupyter Notebook
    \item Google Colab
    \item Marimo
  \end{itemize}
\end{frame}

\begin{frame}[fragile]{How memory\_graph works, simplified}
  \begin{figure}
    \centering
    \includegraphics[width=.8\textwidth]{figures/memory_graph_flow}  
  \end{figure}
  Memory\_Graph is open source, see source for details.
\end{frame}

\begin{frame}{Memory\_Graph Configuration}
  Control what is shown in the graph and how:
  \href{https://memory-graph.com/\#codeurl=https://raw.githubusercontent.com/bterwijn/memory_graph/refs/heads/main/src/config.py}{Configuration Examples}
\end{frame}
  

\begin{frame}{Future Work: Debugging in Production Code}
  \begin{columns}[T]
    \begin{column}[T]{0.75\textwidth}
      Debugging Aid for Production Code?
      \begin{itemize}
      \item debug prints don't show aliasing
      \item debugger tools don't show aliasing
      \end{itemize}
      Use memory\_graph to understand your program state!
      
      \vspace{2em}
      Production Code: But graph gets too big
      \begin{itemize}
      \item control the size of the graph programmatically (config)
      \item future work: interactive graph viewer (similar to \href{https://github.com/jrfonseca/xdot.py}{xdot})
      \end{itemize}
      
      \vspace{2em}
      future work: other languages? \\
      (C/C++, C\#, Java, JavaScript, Go, Rust, ...) 

    \end{column}
    \begin{column}[T]{0.2\textwidth}
      \begin{figure}[t]
        \includegraphics[width=0.9\textwidth]{figures/mutable.png}
      \end{figure}
    \end{column}
  \end{columns}
\end{frame}

\begin{frame}{The End}
  Thanks for your participation. \\
  Thanks Kayode Abiodun Oladapo, for the invitation. \\
  \\
  Any Questions?
\end{frame}

\end{document}
